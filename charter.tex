\documentclass[11pt]{charter}

% El títulos de la memoria, se usa en la carátula y se puede usar el cualquier lugar del documento con el comando \ttitle
\titulo{Mantenimiento predictivo usando cómputo en la nube} 

% Nombre del posgrado, se usa en la carátula y se puede usar el cualquier lugar del documento con el comando \degreename
%\posgrado{Carrera de Especialización en Sistemas Embebidos} 
\posgrado{Carrera de Especialización en Internet de las Cosas} 
%\posgrado{Carrera de Especialización en Intelegencia Artificial}
%\posgrado{Maestría en Sistemas Embebidos} 
%\posgrado{Maestría en Internet de las cosas}

% Tu nombre, se puede usar el cualquier lugar del documento con el comando \authorname
\autor{Juan Pablo Pilorget} 

% El nombre del director y co-director, se puede usar el cualquier lugar del documento con el comando \supname y \cosupname y \pertesupname y \pertecosupname
\director{Yoel López}
\pertenenciaDirector{FIUBA} 
% FIXME:NO IMPLEMENTADO EL CODIRECTOR ni su pertenencia
\codirector{} % si queda vacio no se deberíá incluir 
\pertenenciaCoDirector{}

% Nombre del cliente, quien va a aprobar los resultados del proyecto, se puede usar con el comando \clientename y \empclientename
\cliente{Ariel Lutenberg}
\empresaCliente{Carrera de Especialización en IoT - FIUBA}

% Nombre y pertenencia de los jurados, se pueden usar el cualquier lugar del documento con el comando \jurunoname, \jurdosname y \jurtresname y \perteunoname, \pertedosname y \pertetresname.
\juradoUno{Nombre y Apellido (1)}
\pertenenciaJurUno{pertenencia (1)} 
\juradoDos{Nombre y Apellido (2)}
\pertenenciaJurDos{pertenencia (2)}
\juradoTres{Nombre y Apellido (3)}
\pertenenciaJurTres{pertenencia (3)}
 
\fechaINICIO{25 de agosto de 2020}		%Fecha de inicio de la cursada de GdP \fechaInicioName
\fechaFINALPlanificacion{13 de octubre de 2020} 	%Fecha de final de cursada de GdP
\fechaFINALTrabajo{1 de julio de 2021}		%Fecha de defensa pública del trabajo final


\begin{document}

\maketitle
\thispagestyle{empty}
\pagebreak


\thispagestyle{empty}
{\setlength{\parskip}{0pt}
\tableofcontents{}
}
\pagebreak


\section{Registros de cambios}
\label{sec:registro}


\begin{table}[ht]
\label{tab:registro}
\centering
\begin{tabularx}{\linewidth}{@{}|c|X|c|@{}}
\hline
\rowcolor[HTML]{C0C0C0} 
Revisión & \multicolumn{1}{c|}{\cellcolor[HTML]{C0C0C0}Detalles de los cambios realizados} & Fecha      \\ \hline
1.0      & Creación del documento                                          & 28/08/2020 \\ \hline
1.1      &  Actualización de secciones 1 a 3                           & 04/09/2020 \\ \hline
1.2      & Actualización de secciones 4 a 6                            & 06/09/2020 \\ \hline
1.3      & Actualización de sección 5                            & 13/09/2020 \\ \hline
1.4      & Actualización de secciones 7 a 11                            & 21/09/2020 \\ \hline
\end{tabularx}
\end{table}

\pagebreak



\section{Acta de constitución del proyecto}
\label{sec:acta}

\begin{flushright}
Buenos Aires, \fechaInicioName
\end{flushright}

\vspace{2cm}

Por medio de la presente se acuerda con el Ing. \authorname\hspace{1px} que su Trabajo Final de la \degreename\hspace{1px} se titulará ``\ttitle'', consistirá esencialmente en el desarrollo de una arquitectura en la nube para detección temprana de anomalías en información proveniente de sensores, y tendrá un presupuesto preliminar estimado de 600 hs de trabajo y USD 21.250, con fecha de inicio \fechaInicioName\hspace{1px} y fecha de presentación pública \fechaFinalName.

Se adjunta a esta acta la planificación inicial.

\vfill

% Esta parte se construye sola con la información que hayan cargado en el preámbulo del documento y no debe modificarla
\begin{table}[ht]
\centering
\begin{tabular}{ccc}
\begin{tabular}[c]{@{}c@{}}Ariel Lutenberg \\ Director posgrado FIUBA\end{tabular} & \hspace{2cm} & \begin{tabular}[c]{@{}c@{}}\clientename \\ \empclientename \end{tabular} \vspace{2.5cm} \\ 
\multicolumn{3}{c}{\begin{tabular}[c]{@{}c@{}} \supname \\ Director del Trabajo Final\end{tabular}} \vspace{2.5cm} \\
%\begin{tabular}[c]{@{}c@{}}\jurunoname \\ Jurado del Trabajo Final\end{tabular}     &  & \begin{tabular}[c]{@{}c@{}}\jurdosname\\ Jurado del Trabajo Final\end{tabular}  \vspace{2.5cm}  \\
%\multicolumn{3}{c}{\begin{tabular}[c]{@{}c@{}} \jurtresname\\ Jurado del Trabajo Final\end{tabular}} \vspace{.5cm}                                                                     
\end{tabular}
\end{table}




\section{Descripción técnica-conceptual del proyecto a realizar}
\label{sec:descripcion}

\begin{consigna}{black}
La integración cada vez mayor de la tecnología operacional al mundo IoT plantea desafíos y numerosas oportunidades de desarrollo. 

Una de las nuevas oportunidades es la relacionada a la mejora en la detección y prevención de mantenimiento de los equipos. Los procesos habituales de detección de fallas están vinculados a fallas concretas en el equipamiento o a controles de rutina, con un esfuerzo operacional y un costo económico elevado. Al incorporar capacidades de analítica avanzada no sólo se vuelve más efectivo el proceso -reduciendo los tiempos muertos generados por los controles de rutina- sino que además disminuye el deterioro de los equipos al realizar el mantenimiento de forma preventiva. 

A la vez, la capacidad de ejecutar tareas intensivas de cómputo en la nube permiten escalar los recursos de manera elástica, es decir, aprovisionando recursos para las tareas específicas en cada momento, y escalable, esto es, permitiendo ir conectando dispositivos sin estar limitados por la infraestructura de cómputo subyacente.

En la figura 1 se observa la arquitectura de referencia para solucionar el problema propuesto.


\vspace{25px}

\begin{figure}[htpb]
\centering 
\includegraphics[width=.8\textwidth]{./Figuras/diagrama-bloques.png}
\caption{Diagrama en bloques del sistema}
\label{fig:diagBloques}
\end{figure}


\vspace{25px}

El objetivo de la arquitectura propuesta es identificar anomalías en las señales -si nuestro dispositivo de origen es una turbina de viento, un ejemplo de señales pueden ser la velocidad del rotor, la velocidad del viento o la temperatura exterior- emitidas por los controladores lógicos programables (PLC, por su acrónimo en inglés). Para ello, se pretende desarrollar una arquitectura que, en primer lugar, tome la información de los PLC mediante la tecnología de comunicación industrial OPC-UA y la transmita de manera segura. La arquitectura también debe permitir operar fuera de línea, sincronizando con la nube al recuperar la conectividad. Para este punto se pretende usar AWS IoT Greengrass. En caso de querer modelar también la conexión de los PLC y simular información proveniente de, por ejemplo, turbinas de viento, se puede utilizar la versión demo de KepServerEX. 

Posteriormente, se busca modelar los activos y poder monitorearlos para, mediante reglas, realizar procesos analíticos con esa información. El proceso de modelado y estructuración se diseñó utilizando AWS IoT SiteWise y AWS IoT Core y la analítica se desea calcular con AWS IoT Analytics. 

En este punto es donde se dispara el proceso de detección de anomalías. En primer lugar, un modelo de aprendizaje automático -potencialmente uno de tipo Isolation Forest- entrenado en Amazon SageMaker se disponibiliza (mediante containers) y realiza la inferencia para la información estructurada que ingresa desde AWS IoT Analytics. En segundo lugar, se crea una notificación mediante AWS IoT Events si se cumple una determinada regla, por ejemplo, que un conjunto de predicciones esté consistentemente por encima de un umbral específico. Finalmente, la información producida se puede visualizar de manera periódica mediante un tablero de Amazon Quicksight.

\end{consigna}


\section{Identificación y análisis de los interesados}
\label{sec:interesados}

\begin{consigna}{red} 
\begin{table}[ht]
%\caption{Identificación de los interesados}
%\label{tab:interesados}
\begin{tabularx}{\linewidth}{@{}|l|X|X|l|@{}}
\hline
\rowcolor[HTML]{C0C0C0} 
Rol           & Nombre y Apellido & Organización 	& Puesto 	\\ \hline
Auspiciante   &  \authorname                 &        FIUBA      	&  Alumno      	\\ \hline
Cliente       & \clientename      &\empclientename	&     Director 	\\ \hline
Impulsor      &  -                 &         -     	&     -	\\ \hline
Responsable   & \authorname       & FIUBA        	& Alumno 	\\ \hline
Colaboradores & Consultor experto en IoT                  &    Amazon Web Services   	& IoT Subject Matter Expert       	\\ \hline
Orientador    & \supname	      & \pertesupname 	& Director	Trabajo final \\ \hline
Usuario final &   -             &     Empresa        	&        Ingeniero de procesos / operaciones \\ \hline
\end{tabularx}
\end{table}

\end{consigna}



\section{1. Propósito del proyecto}
\label{sec:proposito}

\begin{consigna}{black}
El propósito de este proyecto es mejorar los actuales procesos de detección de necesidad de mantenimiento en equipamiento industrial, reduciendo tiempos y costos e incrementando la productividad. Para ello, se pretende desarrollar un flujo de trabajo que integre de manera completa -punta a punta- desde la captación de eventos en el dispositivo de borde hasta la inferencia mediante aprendizaje automático.
\end{consigna}

\section{2. Alcance del proyecto}
\label{sec:alcance}

\begin{consigna}{black}
El alcance del proyecto contempla el desarrollo de la arquitectura de transmisión de la información desde los PLC hacia la nube, su análisis y visualización para la detección temprana de necesidad de mantenimiento. 

El presente proyecto no incluye la conexión en un ámbito industrial del equipamiento. Dicho flujo se simulará utilizando herramientas como KepServerEX.

\end{consigna}


\section{3. Supuestos del proyecto}
\label{sec:supuestos}

\begin{consigna}{black}
Para el desarrollo del presente proyecto se supone que:

\begin{itemize}
\item Se cuenta con acceso a servicios de cómputo en la nube (en este caso, Amazon Web Services).
\item La simulación de PLC mediante \textit{software} es condición suficiente para demostrar la viabilidad técnica y financiera del proyecto.
\item Se contará con información de dominio suficiente del ámbito industrial a elegir.
\end{itemize}

\end{consigna}

\section{4. Requerimientos}
\label{sec:requerimientos}
\begin{consigna}{black}
\begin{enumerate}
\item Grupo de requerimientos asociados con \textit{hardware}:
	\begin{enumerate}
	\item Debe enviar señales simulando un PLC mediante un dispositivo de tipo Raspberry Pi.
	\item Debe usar un protocolo de tipo OPC-UA.
	\item Debe recolectar y procesar la información en la nube.
	\item Debe realizar inferencias y permitir visualizarlas en la nube.
	\item Debe almacenar la información de los PLC en la nube.
	\end{enumerate}
\item Grupo de requerimientos asociados con \textit{software}:
	\begin{enumerate}
	\item Debe dirigir la información recibida desde los PLC y redirigida a la nube mediante un motor de reglas.
	\item Debe incluir un modelo de aprendizaje automático de detección de anomalías.
	\item Debe presentar los resultados del modelo de aprendizaje automático detectados como anómalos en \textit{near real-time} mediante un tablero.
	\item Debe generar notificaciones (a ser enviadas a un correo electrónico o mediante SMS).
	\end{enumerate}
\end{enumerate}

\end{consigna}

\section{Historias de usuarios (\textit{Product backlog})}
\label{sec:backlog}

\begin{consigna}{black}
Las historias de usuario, priorizadas y con sus respectivos \textit{story points}, son las siguientes.

\begin{itemize}
\item Como arquitecto de datos quiero desarrollar un diseño detallado de la arquitectura a construir para cumplir con los requerimientos del proyecto. \textit{Story points: 3}
\item Como científico de datos quiero construir, entrenar y poner en producción los modelos de analítica avanzada para eficiencia operacional utilizando cómputo en la nube. \textit{Story points: 7}
\item Como arquitecto de datos quiero construir una flujo de procesamiento para la ingestión, almacenamiento y procesamiento de datos en tiempo real. \textit{Story points: 10}
\item Como ingeniero de operaciones quiero ser capaz de monitorear en tiempo real los indicadores operacionales clave mediante un tablero. \textit{Story points: 3}
\item Como arquitecto de infraestructura quiero construir un vínculo con la red corporativa mediante protocolos estándar para comunicarme con la planta y obtener los datos de los dispositivos existentes. \textit{Story points: 7}
\item Como líder técnico del proyecto quiero construir un listado exhaustivo de épicas, historias de usuario, tareas y su respectivo sprint plan. \textit{Story points: 1}
\item Como ingeniero de operaciones quiero monitorear y gestionar las alarmas, alertas y notificaciones para las acciones. \textit{Story points: 1}

\item Como líder técnico quiero tener la capacidad de automatizar el proceso de construcción de infraestructura mediante código para reducir los tiempos de despliegue. \textit{Story points: 3}
\end{itemize}

%Descripción: En esta sección se deben incluir las historias de usuarios y su ponderación (\textit{story points}). Recordar que las historias de usuarios son descripciones cortas y simples de una característica contada desde la perspectiva de la persona que desea la nueva capacidad, generalmente un usuario o cliente del sistema. La ponderación es un número entero que representa el tamaño de la historia comparada con otras historias de similar tipo.
\end{consigna}

\section{5. Entregables principales del proyecto}
\label{sec:entregables}

\begin{consigna}{black}
Los entregables principales del proyecto serán: 
\begin{itemize}
\item Documentación describiendo el paso a paso de despliegue de la solución y los artefactos asociados (\textit{runbook})
\item Documentación detallando los pasos para la operación de la solución (\textit{playbook})
\item Artefactos de infraestructura como código para la construcción de la arquitectura
\item Código fuente de los modelos de aprendizaje automático desarrollados
\item Informe final de presentación de resultados

\end{itemize}

\end{consigna}

\section{6. Desglose del trabajo en tareas}
\label{sec:wbs}
Todas las tareas se estiman en múltiplos de días completos (jornadas de 8 horas) para llegar al siguiente conjunto de tareas:
\begin{consigna}{black}
\begin{enumerate}

\item \textbf{Planificación del proyecto (48 hs)}
	\begin{enumerate}
	\item Definición del plan de proyecto (24 hs)
	\item Desglose de tareas del plan  (16 hs)
	\item Aprobación del plan (8 hs)
	\end{enumerate}

\item \textbf{Investigación preliminar (72 hs)}
	\begin{enumerate}
	\item Análisis de casos de uso similares (16 hs)
	\item Revisión de documentación de servicios propuestos (32 hs)
	\item Evaluación de alternativas de simulación de datos (16 hs)
	\item Recopilación de código y documentación de soluciones (8 hs)
	\end{enumerate}

\item \textbf{Prueba de concepto (112 hs)}
	\begin{enumerate}
	\item Creación de cuentas y definición de permisos (16 hs)
	\item Compra de equipamiento (8 hs)
	\item Desarrollo de arquitectura (40 hs)
	\item Construcción de \textit{wireframe} (40 hs)
	\item Evaluación de funcionamiento (8 hs)
	\end{enumerate}
	
\item \textbf{Implementación (168 hs)}
	\begin{enumerate}
	\item Creación de representación virtual de datos validados (40 hs)
	\item Recolección e ingesta en la nube de los datos (32hs)
	\item Definición de monitoreo de parámetros (8 hs)
	\item Desarrollo de modelo de analítica avanzada (32hs)
	\item Construccion de tablero de control (24 hs)
	\item Confección de notificaciones y alertas (16 hs)
	\item Creación de plantilla de infraestructura como código (16 hs)
	\end{enumerate}

\item \textbf{Validación (72 hs)}
	\begin{enumerate}
	\item Evaluación de desempeño con datos en tiempo real (32hs)
	\item Realización de pruebas UAT para interfaz (24 hs)
	\item Evaluación de desempeño de alertas y notificaciones (16 hs)
	\end{enumerate}

\item \textbf{Documentación y presentación (128 hs)}
	\begin{enumerate}
	\item Confección de la memoria (40 hs)
	\item Confección del \textit{playbook} (32 hs)
	\item Confección del \textit{runbook} (32 hs)
	\item Desarrollo de la presentación (16 hs)
	\item Presentación (8 hs)
	\end{enumerate}
\end{enumerate}

Cantidad total de horas: 600 hs

\end{consigna}

\section{7. Diagrama de Activity On Node}
\label{sec:AoN}
\begin{consigna}{black}
%La figura \ref{fig:AoN} fue elaborada con el paquete latex tikz y pueden consultar la siguiente referencia \textit{online}:

%\url{https://www.overleaf.com/learn/latex/LaTeX_Graphics_using_TikZ:_A_Tutorial_for_Beginners_(Part_3)\%E2\%80\%94Creating_Flowcharts}
A continuación se presenta el diagrama de \textit{Activity on Node} para el proyecto, coloreado según los grupos de tareas definidas en el apartado inmediato anterior. 
\end{consigna}

\begin{figure}[htpb]
\centering 
\includegraphics[width=.9\textwidth]{./Figuras/AoN.png}
\caption{Diagrama \textit{Activity on Node}}
\label{fig:AON}
\end{figure}


\section{8. Diagrama de Gantt}
\label{sec:gantt}
A continuación se presenta el diagrama de Gantt con las fechas preliminares a revisar y alinear a la fecha de cierre y presentación definida en el acta constitutiva.

\begin{figure}[htpb]
\centering 
\includegraphics[width=1\textwidth]{./Figuras/gantt.png}
\caption{Diagrama de Gantt}
\label{fig:Gantt}
\end{figure}


\section{9. Matriz de uso de recursos de materiales}
\label{sec:recursos}
Para cada tarea desagregada se asigna una estimación de recusos -en horas- y se las agrupa según código de WBS (grupos de tareas según propósito), como se detalla en la siguiente tabla. 
\begin{table}
\label{tab:recursos}
\centering
\begin{tabularx}{\linewidth}{@{}|c|c|c|c|c|@{}}
\hline
\cellcolor[HTML]{C0C0C0} & \cellcolor[HTML]{C0C0C0} & \multicolumn{3}{c|}{\cellcolor[HTML]{C0C0C0}Recursos (hs)} \\ \cline{3-5} 
\multirow{-2}{*}{\cellcolor[HTML]{C0C0C0}\begin{tabular}[c]{@{}c@{}}Código\\ WBS\end{tabular}} & \multirow{-2}{*}{\cellcolor[HTML]{C0C0C0}\begin{tabular}[c]{@{}c@{}}Nombre \\ tarea\end{tabular}} & PC & HW & Rev\\ \hline
Planificación del proyecto & Definición del plan de proyecto & 24 &  &  \\ \hline
Planificación del proyecto & Desglose de tareas del plan &  &  & 16  \\ \hline
Planificación del proyecto & Aprobación del plan &  &  & 8 \\ \hline
Investigación preliminar & Análisis de casos de uso similares 16 &  &  &  \\ \hline
Investigación preliminar & Evaluación de alternativas de simulación & 16  &  &  \\ \hline
Investigación preliminar & Revisión de documentación de servicios &  32 &  &  \\ \hline
Investigación preliminar & Recopilación de código y documentación & 8 &  &  \\ \hline
Prueba de concepto & Creación de cuentas y definición de permisos & 16 &  &  \\ \hline 
Prueba de concepto & Compra de equipamiento &  & 8 &  \\ \hline
Prueba de concepto & Desarrollo de arquitectura & 20 & 20 &  \\ \hline
Prueba de concepto & Construcción de \textit{wireframe} & 30 & 10 &  \\ \hline
Prueba de concepto & Evaluación de funcionamiento &  &  & 8 \\ \hline
Implementación & Creación de representación virtual de datos &  20 & 20 &  \\ \hline
Implementación & Recolección de datos e ingesta en la nube & 24 & 8  &  \\ \hline
Implementación & Definición de monitoreo de parámetros & 8 &  &  \\ \hline
Implementación & Desarrollo de modelo de analítica avanzada & 32 &  &  \\ \hline
Implementación & Construcción de tablero de control & 24 &  &  \\ \hline
Implementación & Confección de notificaciones y alertas & 16 &  &  \\ \hline
Implementación & Creación de plantilla de IaC &  & 16 &  \\ \hline
Validación & Evaluación de desempeño en tiempo real & 32 &  &  \\ \hline
Validación & Realización de pruebas UAT para interfaz & 24 &  &  \\ \hline
Validación & Evaluación de desempeño de alertas & 16 &  &  \\ \hline
Documentación y pres. & Confección de la memoria & 32 &  & 8 \\ \hline
Documentación y pres.& Confección del playbook & 32 &  &  \\ \hline 
Documentación y pres.& Confección del runbook & 32 &  &  \\ \hline
Documentación y pres.& Desarrollo de la presentación & 8 &  & 8  \\ \hline
Documentación y pres.& Presentación & 4 &  & 4 \\ \hline
\end{tabularx}%
\end{table}

{\footnotesize
Referencias:
\begin{itemize}
	\item PC = Investigación, redacción y desarrollo de software
	\item HW = Compra y desarrollo de hardware
	\item Rev = Revisión de terceros
\end{itemize}
} %footnotesize


\section{10. Presupuesto detallado del proyecto}
\label{sec:presupuesto}

\begin{consigna}{black}

\begin{table}[htpb]
\centering
\begin{tabularx}{\linewidth}{@{}|X|c|r|r|@{}}
\hline
\rowcolor[HTML]{C0C0C0} 
\multicolumn{4}{|c|}{\cellcolor[HTML]{C0C0C0}COSTOS DIRECTOS} \\ \hline
\rowcolor[HTML]{C0C0C0} 
Descripción &
  \multicolumn{1}{c|}{\cellcolor[HTML]{C0C0C0}Cantidad} &
  \multicolumn{1}{c|}{\cellcolor[HTML]{C0C0C0}Valor unitario} &
  \multicolumn{1}{c|}{\cellcolor[HTML]{C0C0C0}Valor total} \\ \hline
Horas de desarrollo & 
  \multicolumn{1}{c|}{600} & 
  \multicolumn{1}{c|}{25} & 
  \multicolumn{1}{c|}{15000} \\ \hline
Horas de cómputo en la nube &
  \multicolumn{1}{c|}{1200} & 
  \multicolumn{1}{c|}{1,5} & 
  \multicolumn{1}{c|}{1800} \\ \hline
  Raspberry Pi y periféricos & 
  \multicolumn{1}{c|}{1} & 
  \multicolumn{1}{c|}{150} & 
  \multicolumn{1}{c|}{150} \\ \hline
Materiales varios &
  \multicolumn{1}{c|}{1} & 
  \multicolumn{1}{c|}{50} & 
  \multicolumn{1}{c|}{50} \\ \hline

\multicolumn{3}{|c|}{SUBTOTAL} &
  \multicolumn{1}{c|}{} \\ \hline
\rowcolor[HTML]{C0C0C0} 
\multicolumn{4}{|c|}{\cellcolor[HTML]{C0C0C0}COSTOS INDIRECTOS} \\ \hline
\rowcolor[HTML]{C0C0C0} 
Descripción &
  \multicolumn{1}{c|}{\cellcolor[HTML]{C0C0C0}Cantidad} &
  \multicolumn{1}{c|}{\cellcolor[HTML]{C0C0C0}Valor unitario} &
  \multicolumn{1}{c|}{\cellcolor[HTML]{C0C0C0}Valor total} \\ \hline
\multicolumn{1}{|l|}{25\% del costo directo (hs de revisión de terceros)} &
   1 & 
   4250 &
  4250  \\ \hline
\multicolumn{1}{|l|}{} &
   &
   &
   \\ \hline
\multicolumn{3}{|c|}{SUBTOTAL} &
  \multicolumn{1}{c|}{21250} \\ \hline
\rowcolor[HTML]{C0C0C0}
\multicolumn{3}{|c|}{TOTAL} &
 21250  \\ \hline
\end{tabularx}%
\end{table}

\end{consigna}

\section{11. Matriz de asignación de responsabilidades}
\label{sec:responsabilidades}
\begin{consigna}{black}
La asignación de responsabilidades y el manejo de la autoridad del proyecto se gestionará de la siguiente manera:

\begin{table}[htpb]
\centering
\resizebox{\textwidth}{!}{%
\begin{tabular}{|c|c|c|c|c|c|}
\hline
\rowcolor[HTML]{C0C0C0} 
\cellcolor[HTML]{C0C0C0} &
  \cellcolor[HTML]{C0C0C0} &
  \multicolumn{4}{c|}{\cellcolor[HTML]{C0C0C0}Nombres y roles del proyecto} \\ \cline{3-6} 
\rowcolor[HTML]{C0C0C0} 
\cellcolor[HTML]{C0C0C0} &
  \cellcolor[HTML]{C0C0C0} &
  Responsable &
  Orientador &
  Equipo &
  Cliente \\ \cline{3-6} 
\rowcolor[HTML]{C0C0C0} 
\multirow{-3}{*}{\cellcolor[HTML]{C0C0C0}\begin{tabular}[c]{@{}c@{}}Código\\ WBS\end{tabular}} &
  \multirow{-3}{*}{\cellcolor[HTML]{C0C0C0}Nombre de la tarea} &
  \authorname &
  \supname &
  Experto en IoT &
  \clientename \\ \hline
Planificación del proyecto & Definición del plan de proyecto & P & A & & C \\ \hline
Planificación del proyecto & Desglose de tareas del plan & P & I &  & I  \\ \hline
Planificación del proyecto & Aprobación del plan & P & A & & I \\ \hline
Investigación preliminar & Análisis de casos de uso similares & P & I  & C &  \\ \hline
Investigación preliminar & Evaluación de alternativas de simulación & P  &  & C & I \\ \hline
Investigación preliminar & Revisión de documentación de servicios &  P &  & I &  \\ \hline
Investigación preliminar & Recopilación de código y documentación & P & & C &  \\ \hline
Prueba de concepto & Creación de cuentas y definición de permisos & P & & & I \\ \hline 
Prueba de concepto & Compra de equipamiento & P & & C &  \\ \hline
Prueba de concepto & Desarrollo de arquitectura & P & A & C &  \\ \hline
Prueba de concepto & Construcción de \textit{wireframe} & P & A &  & A  \\ \hline
Prueba de concepto & Evaluación de funcionamiento & P & I &  & I \\ \hline
Implementación & Creación de representación virtual de datos &  P & & C &  \\ \hline
Implementación & Recolección de datos e ingesta en la nube & P &  &  &  \\ \hline
Implementación & Definición de monitoreo de parámetros & P &  & & C  \\ \hline
Implementación & Desarrollo de modelo de analítica avanzada & P &  & C & A  \\ \hline
Implementación & Construcción de tablero de control & P & & C & A \\ \hline
Implementación & Confección de notificaciones y alertas & P & &  & A \\ \hline
Implementación & Creación de plantilla de IaC & P & & C & I \\ \hline
Validación & Evaluación de desempeño en tiempo real & P & I & & A \\ \hline
Validación & Realización de pruebas UAT para interfaz & P & I & & A \\ \hline
Validación & Evaluación de desempeño de alertas & P & I &  & A \\ \hline
Documentación y pres. & Confección de la memoria & P & A &  &  \\ \hline
Documentación y pres.& Confección del playbook & P & I &  & A \\ \hline 
Documentación y pres.& Confección del runbook & P & I & & A \\ \hline
Documentación y pres.& Desarrollo de la presentación & P & A & &  \\ \hline
Documentación y pres.& Presentación & P & A &  & \\ \hline
\end{tabular}%
}
\end{table}


{\footnotesize
Referencias:
\begin{itemize}
	\item P = Responsabilidad Primaria
	\item S = Responsabilidad Secundaria
	\item A = Aprobación
	\item I = Informado
	\item C = Consultado
\end{itemize}
} %footnotesize

%Una de las columnas debe ser para el Director, ya que se supone que participará en el proyecto.
%A su vez se debe cuidar que no queden muchas tareas seguidas sin ``A'' o ``I''.

\end{consigna}

\section{12. Gestión de riesgos}
\label{sec:riesgos}

\begin{consigna}{black}
a) Los riesgos y sus consecuencias son:
 
Riesgo 1: que no se pueda utilizar KepServerEX en la Raspberry Pi
\begin{itemize}
\item Severidad (S): 6. Impacta en la posibilidad de simular el tráfico.
\item Probabilidad de ocurrencia (O): 6. Es una posibilidad que sólo se pueda utilizar en entornos Windows.

\end{itemize}   

Riesgo 2: que no sea posible entrenar el modelo por falta de cantidad o calidad de información
\begin{itemize}
\item Severidad (S): 9. Imposibilita la realización de mantenimiento predictivo.
\item Ocurrencia (O): 3. Los modelos no supervisados pueden trabajar con un conjunto acotado de datos y se puede incrementar la tasa de muestreo para tener más puntos sobre los que trabajar.
\end{itemize}

Riesgo 3: que no sea posible realizar pruebas UAT por falta de disponibilidad del cliente
\begin{itemize}
\item Severidad (S): 7. Imposibilitaría tener una acerca de la de la solución.
\item Ocurrencia (O): 5. El cliente será informado y podrá observar el avance mediante \textit{demos} de manera periódica.
\end{itemize}

Riesgo 4: que no sea posible enviar las alertas vía SMS
\begin{itemize}
\item Severidad (S): 5. Existe la posibilidad de enviar las alertas por correo electrónico.
\item Ocurrencia (O): 2. Los sistemas de envío de mensajes de texto de AWS están probados y son robustos.
\end{itemize}

Riesgo 5: Que no sea posible actualizar el tablero en \textit{near real-time}.
\begin{itemize}
\item Severidad (S): 5
\item Ocurrencia (O): 5
\end{itemize}

b) Tabla de gestión de riesgos:      (El RPN se calcula como RPN=SxO)

\begin{table}[htpb]
\centering
\begin{tabularx}{\linewidth}{@{}|X|c|c|c|c|c|c|@{}}
\hline
\rowcolor[HTML]{C0C0C0} 
Riesgo & S & O & RPN & S* & O* & RPN* \\ \hline
    1   &  6 &  6 & 36    &  7  &  2  & 14     \\ \hline
    2  &  9 &  3 &   27  &    &    &      \\ \hline
    3  &  7 &  5 &   35  &  8  &  2  & 16     \\ \hline
    4  &  5 &  2 &   14  &    &    &      \\ \hline
    5  &  5 &  5 &   25  &    &    &      \\ \hline
\end{tabularx}%
\end{table}

Criterio adoptado: 
Se tomarán medidas de mitigación en los riesgos cuyos números de RPN sean mayores a 30

c) Plan de mitigación de los riesgos que originalmente excedían el RPN máximo establecido:
 
Riesgo 1: los procesos de recolección de casos similares permiten contar con herramientas similares que corran en entornos Linux  

  - Severidad (S): 7. Implicaría simular datos mediante equipamiento físico.
  
  - Probabilidad de ocurrencia (O): 2. Existen alternativas para simular equipamiento físico entre las que se identificó, por ejemplo, una que simula un PLC Beckhoff en Linux.

Riesgo 3: el cliente irá observando el análisis de la solución iterativamente, partiendo de un \textit{wireframe}, lo que permitirá reducir la criticidad de UAT

- Severidad: 8. Si el cliente no logra participar de la validación iterativa no existe ningún tipo de retroalimentación sobre la solución.

- Ocurrencia: 2. Se disponen de instancias quincenales para detectar tempranamente la falta de disponibilidad del cliente y proponer alternativas de validación.
 
\end{consigna}


\section{13. Gestión de la calidad}
\label{sec:calidad}
\begin{consigna}{black}
Acciones de gestión de calidad para requerimientos asociados con \textit{hardware}:

	\begin{enumerate}
	\item Debe enviar señales simulando un PLC mediante un dispositivo de tipo Raspberry Pi.
	\begin{itemize}
	\item Verificación: garantizar que la Raspberry Pi pueda correr un agente tipo KepServerEX para simular tráfico. 
	\item Validación: realizar prueba de compatibilidad de software y simulación de tráfico.
	\end{itemize}
	
	\item Debe usar un protocolo de tipo OPC-UA.
		\begin{itemize}
	\item Verificación: asegurarse de que el software permite simular tráfico de protocolo OPC-UA en una Raspberry.
	\item Validación: verificar en la documentación de KepServerEX la capacidad de simular tráfico  de protocolo OPC-UA.
	\end{itemize}

	\item Debe recolectar y procesar la información en la nube.
		\begin{itemize}
	\item Verificación: verificar que el tráfico de la Raspberry ingrese correctamente al servicio de almacenamiento por objetos de Amazon Web Services.
	\item Validación: revisar que el agente de envío de datos está instalado en el cliente y verificar la carga de los datos.
	\end{itemize}

	\item Debe realizar inferencias y permitir visualizarlas en la nube.
		\begin{itemize}
	\item Verificación: garantizar que el modelo, ante un dato entrante, produzca un resultado de un \textit{score} de anomalía.
	\item Validación: enviar al servicio que realiza la inferencia un registro mediante una llamada de un servicio tipo Postman y ver su respuesta.
	\end{itemize}

	\item Debe almacenar la información de los PLC en la nube.
		\begin{itemize}
	\item Verificación: asegurarse de que la información enviada por la simulación de PLC en la Raspberry Pi esté siendo persistida en la nube.
	\item Validación: verificar el status del envío de registros (logs) de los PLC en el servicio de almacenamiento de Amazon Web Services.
	\end{itemize}
	\end{enumerate}

Acciones de gestión de calidad para requerimientos asociados con \textit{software}:
	\begin{enumerate}
	\item Debe dirigir la información recibida desde los PLC y redirigida a la nube mediante un motor de reglas.
	\begin{itemize}
	\item Verificación: asegurar el funcionamiento del motor de reglas de AWS IoT Core.
	\item Validación: revisar los logs de envío del tráfico al servicio de Analytics de IoT de AWS.
	\end{itemize}
	\item Debe incluir un modelo de aprendizaje automático de detección de anomalías.
	\begin{itemize}
	\item Verificación: asegurar que los registros que ingresan a la nube sean evaluados por el modelo.
	\item Validación: analizar los logs de envío de registros a la nube y los de inferencias realizadas por el servidor que aloje el modelo de analítica avanzada. 
	\end{itemize}
	\item Debe presentar los resultados del modelo de aprendizaje automático detectados como anómalos en \textit{near real-time} mediante un tablero.
	\begin{itemize}
	\item Verificación: el envío de los datos debe tener una demora en su visualización menor a las 12 horas.
	\item Validación: revisar que la tasa de refresco de los datos no sea mayor al umbral establecido.
	\end{itemize}
	\item Debe generar notificaciones (a ser enviadas a un correo electrónico o mediante SMS).
	\begin{itemize}
	\item Verificación: asegurarse de que, ante una anomalía, se envía correctamente una notificación de alerta mediante correo electrónico o SMS.
	\item Validación: simular un dato que el sistema reconoce como anómalo y observar si envía el mensaje por el medio elegido.
	\end{itemize}
	\end{enumerate}

\end{consigna}

\section{14. Comunicación del proyecto}
\label{sec:comunicaciones}

El plan de comunicación del proyecto es el siguiente:

\begin{table}[htpb]
\centering
\begin{tabularx}{\linewidth}{@{}|X|C{2.4cm}|C{3cm}|C{1.8cm}|C{2cm}|C{2.1cm}|@{}}
\hline
\rowcolor[HTML]{C0C0C0} 
\multicolumn{6}{|c|}{\cellcolor[HTML]{C0C0C0}PLAN DE COMUNICACIÓN DEL PROYECTO}           \\ \hline
\rowcolor[HTML]{C0C0C0} 
¿Qué comunicar? & Audiencia & Propósito & Frecuencia & Método de comunicac. & Responsable \\ \hline
Comienzo del proyecto & Director, cliente & Informar sobre el comienzo del proyecto & Única & Email & Alumno, director \\ \hline
Status general & Director & Informar sobre el estado general & Mensual & Email & Alumno  \\ \hline
Ceremonia de cierre de \textit{sprint} & Cliente & Mostrar el avance y planificar & Quincenal  & Email & Alumno  \\ \hline
Revisión técnica & Orientador  & Resolver consultas técnicas específicas & Semanal & Email &  Alumno \\ \hline
Cierre & Orientador, Cliente & Informar sobre la finalización del trabajo y su presentación & Única  & Email & Alumno, Director  \\ \hline
\end{tabularx}
\end{table}

\section{15. Gestión de compras}
\label{sec:compras}

\begin{consigna}{black}

Los costos directos vinculados horas de cómputo en la nube se cubren mediante créditos previamente asignados, por lo que no es necesario gestionar compras.

La compra de otros materiales (Raspberry Pi y periféricos) para la realización del proyecto que figuran como costos directos ya fue realizada, lo que implica que no es necesario emitir un \textit{Statement of Work}.

\end{consigna}

\section{16. Seguimiento y control}
\label{sec:seguimiento}

\begin{consigna}{black}
\end{consigna}

\begin{table}[!htpb]
\centering
%\begin{tabularx}{\linewidth}{@{}|X|X|X|X|X|X|@{}}
\begin{tabularx}{\linewidth}{@{}|C{3cm}|C{2.5cm}|C{2cm}|C{2cm}|C{2cm}|C{2cm}|@{}}
\hline
\rowcolor[HTML]{C0C0C0} 
\multicolumn{6}{|c|}{\cellcolor[HTML]{C0C0C0}SEGUIMIENTO DE AVANCE}                                                                       \\ \hline
\rowcolor[HTML]{C0C0C0} 
Tarea del WBS & Indicador de avance & Frecuencia de reporte & Resp. de seguimiento & Persona a ser informada & Método de comunic. \\ \hline

Planificación del proyecto    &   Cantidad de tareas implementadas        &    Quincenal & \authorname       &    \clientename, \supname        &           Email  \\ \hline
Investigación preliminar &  Cantidad de tareas implementadas &    Quincenal & \authorname       &            \clientename, \supname &                        Email           \\ \hline
Prueba de concepto & Cantidad de tareas implementadas &   Quincenal &\authorname         &           \clientename, \supname &                           Email  \\ \hline
Implementación &   Cantidad de tareas implementadas        &    Quincenal & \authorname       &           \clientename, \supname &                             Email      \\ \hline
Validación                &      Cantidad de tareas implementadas     &    Quincenal &\authorname        &           \clientename, \supname &                             Email      \\ \hline
Documentación y presentación &     Cantidad de tareas implementadas  &    Quincenal &  \authorname , \supname         &     \clientename, \supname       &               Email        \\ \hline

\end{tabularx}%
%}
\end{table}

\section{17. Procesos de cierre}    
\label{sec:cierre}

\begin{consigna}{black}
Siguiendo el enfoque de metodologías ágiles asociado a las historias de usuario detalladas en el apartado correspondiente, se propone como proceso de cierre para cada \textit{sprint} las siguientes actividades:

\begin{itemize}
\item Para cada \textit{sprint} de dos semanas el alumno será responsable de analizar en conjunto con el orientador y el cliente los siguientes puntos:
\begin{itemize}

\item Visualizar lo realizado hasta el momento mediante una \textit{demo} que permita evaluar el grado de avance. 
\item Analizar qué historias de usuario se cumplieron (total o parcialmente) y cómo se relaciona eso con lo planificado al comienzo del \textit{sprint}.
\item Definir en retrospectiva sobre los puntos de mejora y  enunciar tareas concretas a las que se compromete para 	que suceda.
\item Planificar las tareas (vinculadas a historias de usuario) para el siguiente \textit{sprint}. 
\end{itemize}
\item Como cierre del proyecto se realizará una presentación virtual de agradecimiento a cargo del alumno.

\end{itemize}

\end{consigna}


\end{document}
